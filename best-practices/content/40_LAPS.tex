\chapter{\acrfull{laps}}
\section{Einleitung}
Die \acrfull{laps} ist eine Lösung von Microsoft.
Mit \acrshort{laps} werden die Passwörter der lokalen Administratoren, welche auf allen Windows Geräten vorhanden sind, zufällig und unterschiedlich voneinander gesetzt.
Zusätzlich werden die Paswörter in einem gewissen Zeitraum automatisch neu gesetzt.\\

Dies verhindert, dass ein Angreiffer auf weitere Systeme vordringen kann, wenn ein lokales Administratorpasswort kompromittiert ist.
Die Passwörter sind in einem Attribut auf dem Computer Objekt im Active Directory hinterlegt.
Zugriff auf dieses Attribut haben nur berechtigte Benutzer.

\subsection{Voraussetzungen}
Um \acrshort{laps} einsetzen zu können, wird eine Active Directory benötigt mit allen Windows Geräten als Computer Objekte.

\section{Installation}
Die Installation ist in drei Schritte unterteilt.
\begin{enumerate}
    \item Als erstes wird \acrshort{laps} via Group Policy auf allen Windows Geräten installiert.
    \item Danach wird das Active Directory für \acrshort{laps} vorbereitet. Das Active Directory braucht zwei zusätzliche Attribute auf den Computer Objekten um die Passwörter verwalten zu können.
          \begin{itemize}
              \item \textbf{ms-Mcs-AdmPwd}: Speichert das Administrator Passwort in Klartext.
              \item \textbf{ms-Mcs-AdmPwdExpirationTime}: Speichert den Zeitpunkt für den Passwortwechsel.
          \end{itemize}
    \item Zum Schluss wird \acrshort{laps} per Group Policy aktiviert.
\end{enumerate}

\subsection{Installation via \acrshort{gpo}}
\acrshort{laps} kann auf der \href{https://www.microsoft.com/download/details.aspx?id=46899}{Webseite von Microsoft}\footnote{Link: https://www.microsoft.com/download/details.aspx?id=46899} heruntergeladen werden.
Die .msi Datei muss in einem freigegebenen Netzlaufwerk platziert werden, auf welches alle Windows Geräte Zugriff haben.
Zum Beispiel auf dem Domain Controller unter:
\begin{lstlisting}
    C:\Windows\SYSVOL\sysvol\<Domain>
\end{lstlisting}
Dieses Verzeichnis ist Standardmässig auf Domain Controllern freigegeben.\\

Im Group Policy Management muss eine neue Group Policy erstellt werden, welche mit der \acrshort{ou} verknüpft ist, die alle Windows Geräte enthält.
\begin{figure}[H]
    \centering
    \includegraphics[width=0.7\linewidth]{../img/LAPS/GPO-Create-New.png}
    \caption{Neue \acrshort{gpo} für \acrshort{laps} Deployment}
\end{figure}

Mit \textbf{Rechtsklick $\rightarrow$ Edit} kann die neue Group Policy bearbeitet werden.
Unter \textbf{Computer Configuration $\rightarrow$ Policies $\rightarrow$ Software Settings $\rightarrow$ Software installation} kann mit \textbf{Rechtsklick $\rightarrow$ New $\rightarrow$ Package\dots} eine Datei ausgewählt werden, welche installiert werden soll.
Hier muss man die zuvor im freigegebenen Netzlaufwerk abgelegte Installationsdatei auswählen.\\
\begin{minipage}{0.5\linewidth}
    \begin{figure}[H]
        \centering
        \includegraphics[width=\linewidth]{../img/LAPS/GPO-Edit-Deployment.png}
        \caption{\acrshort{gpo} für \acrshort{laps} Deployment bearbeiten}
    \end{figure}
\end{minipage}
\begin{minipage}{0.5\linewidth}
    \begin{figure}[H]
        \centering
        \includegraphics[width=\linewidth]{../img/LAPS/GPO-Edit-Deployment-2.png}
        \caption{\acrshort{laps} Installationsdatei auswählen}
    \end{figure}
\end{minipage}\\

Die Group Policy kann nun geschlossen werden.
\acrshort{laps} wird beim Einloggen auf den jeweiligen Geräten installiert.

\subsubsection{Domain Controller}
Auf dem Domain Controller sollte man \acrshort{laps} manuell installieren.
Bei der manuellen Installation müssen die zusätzlichen Features installiert werden.
Diese sind das \acrshort{gui}, das Powershell Modul und die \acrshort{gpo} Richtlinien.

Dazu die Installationsdatei ausführen und alle Features installieren.
\begin{figure}[H]
    \centering
    \includegraphics[width=0.7\linewidth]{../img/LAPS/laps-ui-install-2.png}
    \caption{\acrshort{laps} manuelle Installation}
\end{figure}


\subsubsection{\acrshort{laps} \acrshort{gui} für Admins}\label{subsubsec:Laps-Gui}
Über die \acrshort{gpo} wird auf den Computern \acrshort{laps} ohne \acrshort{gui} installiert.
Mit dem \acrshort{gui} kann das Passwort von beliebigen Windows Geräten in der Domäne abgefragt werden.\\

Das \acrshort{gui} kann über die Systemeinstellungen installiert werden, wenn \acrshort{laps} bereits installiert ist.
Die Installationsdatei wird dafür nicht benötigt.
In der Programmliste in den Systemeinstellungen auf \acrshort{laps} klicken und im Balken auf \textbf{Change} klicken.
\begin{figure}[H]
    \centering
    \includegraphics[width=0.7\linewidth]{../img/LAPS/laps-ui-install.png}
    \caption{\acrshort{laps} \acrshort{gui} Installieren 1}
\end{figure}

Das \textbf{Management Tools} Feature auswählen und dem Installationsprozess folgen.
\begin{figure}[H]
    \centering
    \includegraphics[width=0.7\linewidth]{../img/LAPS/laps-ui-install-2.png}
    \caption{\acrshort{laps} \acrshort{gui} Installieren 2}
\end{figure}


\subsection{Active Directory vorbereiten}
Die zusätzlichen Extension Attributes können per Powershell hinzugefügt werden.
Der ausführende Domänenbenutzer muss ein ``Schema Admin'' in Active Directory sein.\\
\begin{figure}[H]
    \centering
    \includegraphics[width=0.7\linewidth]{../img/LAPS/schema-admins.png}
    \caption{Schema Admins Gruppe}
\end{figure}

Powershell mit dem Schema Admin Benutzer starten und folgendes Eingeben
\begin{lstlisting}
Import-module AdmPwd.PS
Update-AdmPwdADSchema

#Resultat:
#Operation            DistinguishedName                                              Status
#---------            -----------------                                              ------
#AddSchemaAttribute   cn=ms-Mcs-AdmPwdExpirationTime,CN=Schema,CN=Configuration,DC=b Success
#AddSchemaAttribute   cn=ms-Mcs-AdmPwd,CN=Schema,CN=Configuration,DC=ba,DC=lab       Success
#ModifySchemaClass    cn=computer,CN=Schema,CN=Configuration,DC=ba,DC=lab            Success

Set-AdmPwdComputerSelfPermission -OrgUnit "<Name der OU>"
#Resultat:
#Name                 DistinguishedName                                              Status
#----                 -----------------                                              ------
#Clients              OU=Clients,OU=Computers,OU=Lab,DC=ba,DC=lab                    Delegated
\end{lstlisting}

Nun muss noch eine Active Directory Gruppe erstellt werden.
Dieser Gruppe kann man alle Benutzer hinzufügen, welche Zugriff auf die Passwörter bekommen sollen.

\begin{figure}[H]
    \centering
    \includegraphics[width=\linewidth]{../img/LAPS/Laps-Admins.png}
    \caption{\acrshort{laps} Active Directory Gruppe}
\end{figure}

Nach dem Erstellen der Gruppe, muss zusätzlich noch eine Berechtigung auf der \acrshort{ou} für die Gruppe eingerichtet werden.
Dazu muss ASEdit und die Eigenschaften der \acrshort{ou} geöffnet werden. Im ``Security'' Tab
\begin{figure}[H]
    \centering
    \includegraphics[width=0.9\linewidth]{../img/LAPS/ASEdit.png}
    \caption{ASEdit}
\end{figure}
unter ``Advanced'' muss die neue Gruppe hinzugefügt werden mit der Berechtigung ``All Extended Rights''.\\

Nun muss für \acrshort{laps} noch die Verknüpfung zwischen dieser Gruppe und der \acrshort{ou} erstellt werden.
Dies kann mit Powershell erledigt werden:
\begin{lstlisting}
Set-AdmPwdReadPasswordPermission -OrgUnit "<Name der OU>" -AllowedPrincipals Global_LAPS-Admins
Set-AdmPwdResetPasswordPermission -OrgUnit "<Name der OU>" -AllowedPrincipals Global_LAPS-Admins
\end{lstlisting}
Die Read Berechtigung erlaubt das Lesen der Passwörter.
Die Reset Berechtigung erlaubt das setzen eines Ablaufdatums eines Passwortes.
Die Berechtigung kann nun noch mii Powershell überprüft werden:
\begin{lstlisting}
Find-AdmPwdExtendedrights -identity "<Name der OU>"
\end{lstlisting}


\subsection{\acrshort{laps} aktivieren}
Zum Schluss muss \acrshort{laps} noch mit einer neuen Group Policy aktiviert werden.

Im Group Policy Management muss eine neue Group Policy erstellt werden, welche mit der gleichen \acrshort{ou} verknüpft ist, wie die Group Policy für das Deployment.
\begin{figure}[H]
    \centering
    \includegraphics[width=0.7\linewidth]{../img/LAPS/GPO-Create-New.png}
    \caption{Neue Group Policy für \acrshort{laps} Deployment}
\end{figure}

Mit \textbf{Rechtsklick $\rightarrow$ Edit} kann die neue Group Policy bearbeitet werden.
Unter \textbf{Computer Configuration $\rightarrow$ Policies $\rightarrow$ Administrative Templates $\rightarrow$ Laps} findet man die Einstellungen von \acrshort{laps}.
Mit \textbf{Rechtsklick $\rightarrow$ Edit} auf ``Enable local admin password management'' kann man die Einstellung öffnen und links auf ``Enabled'' setzen.
Mit OK bestätigen.
\begin{figure}[H]
    \centering
    \includegraphics[width=0.7\linewidth]{../img/LAPS/enable-laps.png}
    \caption{\acrshort{laps} aktivieren}
\end{figure}

In der Einstellung ``Password Settings'' kann man die Passwort Länge und den Zyklus setzen, wie oft das Passwort gewechselt werden soll.\\

Zusätzlich muss noch der lokale Administrator aktiviert werden.
Dieser ist standardmässig bei Computern welche der Active Directory beigetreten sind deaktiviert.
Unter \textbf{Computer Configuration $\rightarrow$ Policies $\rightarrow$ Windows Settings $\rightarrow$ Security Settings $\rightarrow$ Local Policies $\rightarrow$ Security Options } findet man die Einstellungen ``Accounts: Administrator Account Settings''.
Diese setzt man auf ``Enabled''.
Die Group Policy kann nun geschlossen werden.


\section{Verwendung}
\subsection{Mit dem \acrshort{gui}}\label{subsec:laps-gui-usage}
Das Auslesen eines Passwortes über das \acrshort{gui} geht nur, wenn das \acrshort{gui} auch installiert wurde.
Die Installation des \acrshort{gui} wird im Kapitel \hyperref[subsubsec:Laps-Gui]{LAPS GUI für Admins} erklärt.\\

Im Startmenü muss man nach \textbf{\acrshort{laps} UI} suchen. Falls der in Windows angemeldete Benutzer Berechtigung besitzt, \acrshort{laps} Passwörter auszulesen, kann \acrshort{laps} direkt gestartet werden.\\
Falls nicht, muss man den Speicherort der Datei öffnen.
Dann kann man mit \textbf{Shift + Rechtsklick} das Kontextmenü öffnen und die Datei mit ``Run as different user'' als anderen Benutzer ausführen.
Ein Eingabefenster öffnet sich, wo man den berechtigten Benutzer und das Passwort des Benutzer eingeben kann.
Berechtigt sind alle Benutzer in der zuvor erstellten Gruppe und Domänen Administratoren.\\
\begin{minipage}{0.5\linewidth}
    \begin{figure}[H]
        \centering
        \includegraphics[width=\linewidth]{../img/LAPS/usage-1.png}
        \caption{\acrshort{laps} Speicherort öffnen}
    \end{figure}
\end{minipage}
\begin{minipage}{0.5\linewidth}
    \begin{figure}[H]
        \centering
        \includegraphics[width=\linewidth]{../img/LAPS/usage-2.png}
        \caption{\acrshort{laps} als anderer Benutzer starten}
    \end{figure}
\end{minipage}\\

Im \acrshort{laps} UI gibt man im Feld ``Computer name'' den Namen des Computers ein, von welchem man das lokale Passwort möchte.
Das Passwort wird dann im Feld ``Password'' angezeigt.
\begin{figure}[H]
    \centering
    \includegraphics[width=0.7\linewidth]{../img/LAPS/Laps-ui.png}
    \caption{\acrshort{laps} Passwort auslesen}
\end{figure}
Im Feld ``New expiration time'' kann man eine neue Zeit setzen, wann das Passwort gewechselt werden soll.

\subsection{Mit Powershell}
Die Powershell muss als Benutzer gestartet werden, welcher die Berechtigung hat das Passwort auszulesen oder zurückzusetzen.
Powershell als ein anderer Benutzer starten kann man gleich wie \acrshort{laps} im Kapitel \hyperref[subsec:laps-gui-usage]{``Mit dem GUI''} als anderer Benutzer gestartet wurde.\\

Für das Auslesen und Zurücksetzen gibt es folgende Befehle:
\begin{lstlisting}
    Get-AdmPwdPassword -ComputerName <Computer Name>
    Reset-AdmPwdPassword -ComputerName <Computer Name>
\end{lstlisting}


\section{Best Practices}
Idealerweise sollten nur ausgewählte IT Mitarbeitende die Berechtigung erhalten, Passwörter auslesen zu können.
Alle lokalen Administratorrechte von AD Benutzern sollten entfernt werden und nur noch \acrshort{laps} Passwörter verwendet werden.
Dadurch kann man verhindern, dass es einen Benutzer gibt, welcher auf allen Systemen berechtigt ist.\\

Falls ein Mitarbeitender ohne \acrshort{laps} Berechtigung lokale Administratorrechte braucht, kann diesem das \acrshort{laps} Paswort gegeben werden.
Dann sollte man aber auch das ``Expire Date'' auf das Datum setzen, wo das Passwort nicht mehr gebraucht wird.
Zum Beispeil am nächsten Tag.

\section{Intergration in Wazuh}
In Wazuh existiert eine Regel, welche jedes auslesen meldet.
Die Regel hat die ID 110010 und der Agent Name ist der \textbf{Domain Controller}.
Es ist nicht der Computer, auf welchem das Passwort ausgelesen wurde.
\begin{figure}[H]
    \centering
    \includegraphics[width=\linewidth]{../img/LAPS/laps-wazuh.png}
    \caption{\acrshort{laps} Wazuh Alert}
\end{figure}
Somit weiss man wer für welchen Computer das Passwort ausgelesen hat.
Man kann jedoch nicht feststellen, von welchem Computer aus das Passwort ausgelesen wurde.\\

Zusätzlich wird nur die GUID des Computers angezeigt, für welchen man das Passwort ausgelesen hat.
Um den Computernamen zu finden, kann man folgenden Powershell Befehl verwenden:
\begin{lstlisting}
    #GUID mit { } eingeben!
    Get-ADObject -Identity "<GUID>"
\end{lstlisting}

Zusätzlich gibt es noch eine Regel, welche einen Alert generiert, sobald 20 \acrshort{laps} Passwörter in 10 Minuten ausgelesen wurden.
\begin{figure}[H]
    \centering
    \includegraphics[width=\linewidth]{../img/LAPS/laps-wazuh-alert.png}
    \caption{\acrshort{laps} Wazuh Alert 2}
\end{figure}