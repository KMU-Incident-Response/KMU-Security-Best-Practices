\chapter{Backup}

\section{Einleitung}
Ein Backup ist eine Sicherungskopie und kann aus Nutzdaten, Systemdaten oder gar ganzen Datenträgerabbildungen bestehen.
Im normalen, operativen Betrieb wird ein Backup selten benötigt.
Ein Backup wird in nicht alltäglichen Situationen, wie Datenverlust (z.B. Missclick) oder Ransomware, von sehr grossem Nutzen.
Trotz des enormen Mehrwertes des Backups geht es in KMUs oft vergessen.
Dies kann schon bei kleinen Vorfällen schwerwiegende wirtschaftliche Folgen herbeiziehen.




\section{Allgemeine Tipps zum Backup}
Bevor dieser Guide ins Detail vom Backup geht, wird noch ein wichtiger Punkt aufgegriffen.\\

\textbf{Das Backup ist gleich sensibel wie die darin enthaltenen Daten.}
Oft als unkritisch klassifiziert -- das Backup sollte gleich wie die Daten darin behandelt werden.
In der Schlussfolgerung, wenn die Daten verschlüsselt werden, sollte dies das Backup auch.
Es sollte verschlüsselt übermittelt werden, verschlüsselt gespeichert und bei sehr kritischen Daten in einer Form verschlüsselt bearbeitet, resp. im Memory gelagert werden.

\section{Backup Plan}
Ein Backup sollte keine zufällige Sache sein.
Ein Backup sollte regelmässig, vollständig und geplant sein.
Alle nachfolgenden Punkte sollten im Backup Plan berrücksichtigt werden.



\subsection{spezifische Daten}
Ein Backup macht in den meisten Fällen Sinn, wenn spezifische Daten existieren.
Backups können schnell zu einem grossen Kostenpunkt werden, denn diese konsumieren enorm viel Speicher.
Daher sollte es vermieden werden unnötige Daten zu backupen.
Es wäre zum Beispiel denkbar wöchentlich ein Full-Backup zu machen und jeden Tag den inkrement.\\


Die nachfolgenden Beispiele visualisieren, welche Geräte / Daten für ein Backup geeignet sind.\\


\textbf{Beispiel 1:}\\
\textbf{Ausgangslage:} Ein Fileshare mit kritischen Daten, von 100 Usern verwendet, wird gebackupt.\\
\textbf{Erläuterung:} Der oben beschriebene Sachverhalt bestätigt, dass hier ein Backup sinnvoll ist. Denn der Ausfall dieses Server könnte für die Firma den produktiven Betrieb beeinträchtigen. Die Daten könnten durch Verlust grossen Sachschaden bedeuten.\\


\textbf{Beispiel 2:}\\
\textbf{Ausgangslage:} Ein Notebook mit einer standard Firmeninstallation wird gebackupt. Der Benutzer liest seine E-Mails und arbeitet auf dem Gerät in Sharepoint.\\
\textbf{Erläuterung:} Dieses Gerät zu backupen macht keinen Sinn da es weder spezifische Daten beherbergt, noch einen speziellen Wert für das Unternehmen bringt. Der Plan bei einem Notfall sollte hier sein das Gerät neu zu installieren.\\

Es ist wichtig seine Mitarbeiter zu schulen, wie diese sich im Umgang mit Daten verhalten sollen.
Möglich wäre eine Regel die besagt alle kritischen Geschäftsdaten werden auf einem Fileshare abgelegt der gebackupt wird.
Damit ist es nicht nötig jeden Client einzeln zu backupen.

\subsection{Aufbewahrungsdauer / Retention Policy}
Bei Datensicherungen ist es wichtig nicht nur die neuste Version zu behalten, da vielleicht diese nicht vollständig ist.
Die Aufbewahrungsdauer muss in der Firma mit dem Management geklärt werden, da es sich direkt auf die Kosten des Backups auswirkt.
Backup Aufbewahrungsdauer können wie folgt aussehen:\\
\begin{table}[H]
    \begin{center}
        \begin{tabular}{l|l|l}
            \hline
            Intervall    & Dauer             & Art von Backup\\
            \hline
            Täglich      & 1 Woche           & Inkrementell\\
            Wöchentlich  & 1 - 4 Wochen      & Full Backup\\
            Monatlich    & 1 Monat - 2 Jahre & Full Backup\\
            Jährlich     & 2 - 10 Jahre      & Fullbackup\\
            \hline
        \end{tabular}
    \end{center}
    \caption{Backup Aufbewahrungsdauer}
\end{table}


\subsection{Datenspeicherort}
Das Backup sollte auch in einem Disaster Fall noch zu Verfügung stehen.
Dies könnte auch eine Naturkatastrophe bedeuten, wie Überschwemmung des Datacenters.
Es wird empfohlen Backups an mehreren geounabhängigen Orten zu lagern.
Ausserdem empfiehlt es sich auch ein Backup Offline zu lagern, im Falle der Komprimitierung des Firmennetzwerkes.
Es wird vermehrt beobachtet, dass Ransomware zuerst das Backup kompromitiert, auch dieser Trend sollte in der Entscheidung des Lagerortes einfliessen.



\subsection{Datenschutz}
Aus Datenschutzgründen muss abgeklärt werden, welche Daten in welchem Land gespeichert werden dürfen.
Dies ist vorallem bei Cloudanbieter kritisch und sollte im vorhinein klar vereinbart werden.
Der Datenspeicherort sollte gegebenfalls mit juristischer Unterstützung geklärt werden.

\section{Emergencyplan}
In einem Notfall ist es wichtig einen ``Emergencyplan'' zu haben.
Die wichtigsten Elemente in einem Notfallplan sind die Prioritäten der Geschäftsprozesse und deren Abhängigkeit an Daten.\\


Das \acrfull{rto} ist die Zeitspanne, innerhalb derer ein Geschäftsprozess nach einer Katastrophe wiederhergestellt werden muss, um unannehmbare Folgen einer Unterbrechung des operativen Betriebs zu vermeiden.\\

Der Notfallplan eines Unternehmens sollte mindestens einmal jährlich validiert werden und im Idealfall ein ähnliches Szenario durch gespielt.
Der Notfallplan sollte mit dem höheren Management des Unternehmens erarbeitet und anschliessend abgesegnet werden.


\section{Wiederherstellen des Backups}
Es scheint trivial, dass ein Backup wiederherstellbar sein sollte.
Leider kann es auch hier technisches Versagen geben.
Es wird stark empfohlen in regelmässigen Abständen das wiederherstellen der Daten zu testen.
Das beste Backup bringt ohne einen funktionierenden Restore nichts.