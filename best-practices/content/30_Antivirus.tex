\chapter{\acrlong{av}}
\section{Allgemeines} %TODO: maybe change title
Ein \acrfull{av} ist der erste Verteidigungsmechanismus in einem Netzwerk. Angriffe, darunter Schadsoftware, kommen in den meinsten Fällen über einen Endpoint (z.B. Notebook) in ein Firmennetzwerk.\\

Eine gegebene Struktur ist immer nur so stabil, wie sein schwächstes Glied. Dies gilt bei Häusern wie auch bei Computernetzwerken.
Oftmals ist das schwächste Glied ein Client oder ein Server, da diese Geräte vielfach mit dem Internet kommunizieren.
Dies macht genau diese Geräte besonders schützenswert.\\

Server \& Clients werden mit einem \acrlong{av} versucht zu schützen. Ein Antivirus hilft bekannte Schadsoftware zu detektieren und entfernen.
Neue Schadsoftware wird versucht mit komplexen Algorithmen zu erkennen und deren Vorhaben zu blockieren.
Schadsoftware kann sich sehr schnell vermehren und sich im ganzen Netzwerk ausbreiten. 
Das kann den Effekt haben, dass die ganze Unternehmung beeinträchtigt wird.
Deshalb ist eines der wichtigesten Erkenntnisse, dass ein \acrlong{av} auf \textbf{allen} Endpoints (Server \& Client) benötigt wird.





\section{Windows Defender}
\acrfull{av} kommen in allen Arten \& Formen.
Es ist schwer den Überblick über die Landschaft der Antiviren Programme zu behalten.\\


Microsoft bietet für seine Produkte einen eigenen kostenfreien \acrlong{av} an.
Der Windows Defender ist nicht eine allround Lösung, wie andere \acrshort{av} Produkte, mit unzähligen Features.
Er ist einer der besten kostenfreien Lösungen \& ist für KMUs sehr geeignet, da dort meist keine massgeschneiderte Lösung gewünscht wird.
Laut dem \href{https://www.microsoft.com/security/blog/2021/05/11/gartner-names-microsoft-a-leader-in-the-2021-endpoint-protection-platforms-magic-quadrant/}{Gartner Magic Quadrant für Endpoint Protection}\footnote{Link: https://www.microsoft.com/security/blog/2021/05/11/gartner-names-microsoft-a-leader-in-the-2021-endpoint-protection-platforms-magic-quadrant/} ist die Microsoft Lösung eine der Marktführer in diesem Bereich.
Der Windows Defender erkennt die meisten Gefahren und löst die erkannten Probleme sehr kompetent.\\

Hier eine Übersicht der Vor- und Nachteile:\\

\begin{minipage}{\linewidth}
    \begin{multicols}{2}
        \begin{table}[H]
            \begin{center}
                \textbf{Vorteile:}
                \begin{itemize}
                    \item kostenfrei mit Windows geliefert
                    \item Realtime Protection funktioniert zuverlässig
                    \item keine Installation nötig
                    \item einfache Bedienung
                \end{itemize}
            \end{center}
            \caption{Vorteile Windows Defender}
        \end{table}
        \begin{table}[H]
            \begin{center}
                \textbf{Nachteile:}
                \begin{itemize}
                    \item beschränkte Konfigurationsmöglichkeiten
                    \item Keine cloudbasiertes Dashboard/Management
                    \item Machine Learning nicht vorhanden
                \end{itemize}
            \end{center}
            \caption{Nachteile Windows Defender}
        \end{table}
    \end{multicols}
\end{minipage}
%TODO: @sevigrimm add more points?




\subsection{Wazuh Integration}
%TODO: @sevigrimm already included correct?