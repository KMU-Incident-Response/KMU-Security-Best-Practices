% in this file
% \newglossaryentry{latex}
% {
%     name=latex,
%     description={Is a arkup language specially suited 
%     for scientific documents}
% }


% In text
% \Gls{latex}

\newglossaryentry{identity-access-management}
{
    name=Identity and Access Management,
    description={Identitäts- und Zugriffsmanagement (IAM) können Administratoren autorisieren, wer auf bestimmte Ressourcen zugreifen darf. So ist es möglich die Kontrolle und Transparenz zentral zu verwalten. Für Unternehmen mit komplexen Organisationsstrukturen, Hunderten von Teams und vielen Projekten bietet IAM eine einheitliche Sicht auf die Sicherheitsrichtlinien in Ihrem gesamten Unternehmen mit integrierter Prüfung zur Vereinfachung der Compliance-Prozesse}
}

\newglossaryentry{azure-active-directory-domain-services}
{
    name=Azure Active Directory Domain Services,
    description={Azure Active Directory (Azure AD) ist ein Cloud-basierter Identitäts- und Zugangsverwaltungsdienst. Dieser Dienst hilft beim Zugriff auf externe Ressourcen, wie Microsoft 365, das Azure-Portal und Tausende anderer SaaS-Anwendungen}
}

\newglossaryentry{network-time-protocol}
{
    name=Network Time Protocol,
    description={Das Network Time Protocol (NTP) wird häufig zur Synchronisierung von Computeruhren im Netzwerk verwendet}
}

\newglossaryentry{certification-authority}
{
    name=Certification Authority,
    description={Die Certification Authoritity (CA) ist eine Stelle, welche digitale Zertifikate ausstellt. Somit ist es möglich bei der Kommunikation zweier Parteien die Integrität durch zuverlässige dritte Partei zu haben}
}

\newglossaryentry{dynamic-host-configuration-protocol}
{
    name=Dynamic Host Configuration Protocol,
    description={Das Dynamic Host Configuration Protocol (DHCP) ist ein Protokoll im Netzwerk. Es ermöglicht die Zuweisung von Netzwerkkonfigurationen, wie IP-Adressen \& Gateway, an Clients durch einen Server}
}

\newglossaryentry{domain-name-system}
{
    name=Domain Name System,
    description={Das Domain Name System (DNS) ist das Telefonbuch eines IP-basierten Netzwerkes. Seine Hauptaufgabe ist die Beantwortung von Anfragen zur Namensauflösung}
}

\newglossaryentry{national-institute-of-standards-and-technology}
{
    name=National Institute of Standards and Technology,
    description={Das National Institute of Standards and Technology ist eine Amerikanische Bundesbehörde, welche für Standardisierungsprozesse zuständig ist}
}

\newglossaryentry{trusted-platform-module}
{
    name=Trusted Platform Module,
    description={Das Trusted Platform Module ist ein optionaler Hardware-Chip auf der Hauptplatine in einem Computer. IM TPM Chip werden Kryptografische Schlüssel hinterlegt, welche dann von Software verwendet werden können}
}

\newglossaryentry{windows-server-update-services}
{
    name=Windows Server Update Services,
    description={Der Windows Server Update Services ist eine Serverrolle für Windows Server mit welcher die Windows Updates zentral Verwaltet werden können. Die Rolle bietet die Möglichkeit, Updates zentral herunterzuladen und auf alle Windows Geräte zu verteilen. Zusätzlich kann genau gestueret werden, welche Computer und Server welche Updates erhalten sollen}
}

\newglossaryentry{common-vulnerabilities-and-exposures}
{
    name=Common Vulnerabilities and Exposures,
    description={Das Common Vulnerabilities and Exposures Referenzier-System wird durch die Mitre Corporation gepflegt und ist dem US National Cybersecurity FFRDC unterstellt. CVEs sind bekannt gewordene Attacken welche dokumentiert und veröffentlich werden. Sie beinhalten die geschätzte Herkunft der Angreiffer, den Angriffsweg und Möglichkeiten, sich gegen solch einen Angriff zu schützten}
}

\newglossaryentry{antivirus}
{
    name=Antivirus,
    description={Ein Antivirus (AV) ist eine Software, die Schadsoftware wie zum Beispiel Viren, Würmer oder Trojanische Pferde aufspüren, blockieren und beseitigen soll}
}

\newglossaryentry{recovery-time-objective}
{
    name=Recovery Time Objective,
    description={Das Recovery Time Objective (RTO) ist die Zeitspanne, innerhalb derer ein Geschäftsprozess nach einer Katastrophe wiederhergestellt werden muss, um unannehmbare Folgen einer Unterbrechung des operativen Betriebs zu vermeiden}
}

\newglossaryentry{organizational-unit}
{
    name=Organizational Unit,
    description={Mit Organizational Units (OUs) in einer von Active Directory (AD) verwalteten Domäne können Objekte wie Benutzeraccounts, Serviceaccounts oder Computer logisch gruppieren}
}

\newglossaryentry{group-policy-object}
{
    name=Group Policy Object,
    description={Ein Group Policy Object (GPO) ist eine Sammlung von Richtlinieneinstellungen. Ein GPO hat einen eindeutigen Namen, z. B. eine GUID. Gruppenrichtlinieneinstellungen sind in einem GPO enthalten}
}

\newglossaryentry{local-administrator-password-solution}
{
    name=Local Administrator Password Solution,
    description={Die Local Administrator Password Solution (LAPS) ermöglicht die Verwaltung der Passwörter lokaler Accounts von Computern, die der Domäne angeschlossen sind. Die Passwörter werden im Active Directory (AD) gespeichert}
}

\newglossaryentry{operating-system}
{
    name=Operating System,
    description={Ein Operatingsystem (OS), auch Betriebssystem genannt, ist eine Systemsoftware, die Computerhardware und Softwareressourcen verwaltet und allgemeine Dienste für Computerprogramme bereitstellt}
}

\newglossaryentry{security-information-and-event-management}
{
    name=Security Information and Event Management,
    description={Security Information and Event Management (SIEM) ist ein Bereich der Informatik, welcher sich mit dem sammeln und auswerten von Logdateien beschäftigt.
    Oftmals wird dies mit SIEM Softwaresystemen gemacht}
}

\newglossaryentry{elasticsearch-logstash-kibana}
{
    name=Elasticsearch Logstash Kibana,
    description={
        Elasticsearch, Logstash, Kibana sind drei Open-Source Projekte. Elasticsearch ist eine Such- und Analysesoftware für Logdateien.
        Logstash ist eine Software für das Sammeln von Logdateien von mehreren Quellen und übergibt diese an einen ``Stash'', zum Beispiel Elasticsearch.
        Kibana ist eine Visualisierungssoftware, welche Grafiken und Diagramme von einem ``Stash'' machen kann
    }
}

\newglossaryentry{graphical-user-interface}
{
    name=Graphical user interface,
    description={
        Ein Graphical user interface, auch Grafische Benutzeroberfläche genannt, ist eine Schnittstelle für Benutzer, um mit einem elektronischen Gerät grafisch zu interagieren
    }
}

\newglossaryentry{kibana-query-language}
{
    name=Kibana query language,
    description={
        Die Kibana query language (KQL) ist eine einfache Abfragesprache, mit welcher man in Text oder Feldbasierter Suche die Elasticsearch Daten durchsuchen kann
    }
}

\newglossaryentry{sans-insitute}
{
    name=SANS Institute,
    description={
        Das SANS (SysAdmin, Audit, Network, and Security) Insitute ist ein private Unternehmen, welches sich auf die Ausbildung und Zertifizierung im Bereich der Cyber Security spezialisiert hat. 
    }
}
