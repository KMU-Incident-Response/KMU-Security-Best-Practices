%! Licence = CC BY-NC-SA 4.0

%! Author = mariuszindel
%! Date = 24. Jan 2021
%! Project = latex-documentation-template


\documentclass[a4paper, oneside, 10pt]{scrreprt}

% charset
\usepackage[T1]{fontenc}
\usepackage[utf8]{inputenc}
\renewcommand{\familydefault}{\sfdefault}
\usepackage[Sonny]{fncychap}
\ChNameVar{\Large\fontfamily{\sfdefault}\bfseries}
\ChNumVar{\Huge\fontfamily{\sfdefault}\bfseries}
\ChTitleVar{\Large\fontfamily{\sfdefault}\bfseries}

% use language german
\usepackage[ngerman]{babel}

% format page size
\usepackage{geometry}
\geometry{top=2.54cm,left=2.54cm,right=2.54cm}

% tabular
\usepackage{tabularx}
\usepackage{multirow}
\usepackage{colortbl}

% math
\usepackage{amsmath}
\usepackage{amssymb}
\usepackage{amsfonts}
\usepackage{enumitem}

% graphic
\usepackage{graphicx}
\graphicspath{{img/}}
\usepackage{caption}
\usepackage{float}

% colors
\usepackage[dvipsnames]{xcolor}

% multi columns
\usepackage{multicol}

% Enable clickable table of contents
\usepackage{hyperref}
\hypersetup{
    colorlinks,
    linkcolor={black},
    citecolor={blue!50!black},
    urlcolor={blue!80!black}
}

% make items compact
%\setlist{topsep=0pt, leftmargin=4mm, nolistsep}
%\setlength{\parindent}{0cm}

% author and institute
\newcommand{\AUTHORONE}{Severin Grimm}
\newcommand{\AUTHORTWO}{Marco Martinez}
\newcommand{\INSTITUTE}{Ostschweizer Fachhochschule}

% define header and footer
\usepackage{fancyhdr}
\pagestyle{fancy}
\setkomafont{pagehead}{\normalfont}
\setkomafont{pagefoot}{\normalfont}

\fancyhead[LO]{\nouppercase \leftmark}
\fancyhead[RO]{\AUTHORONE\space \& \AUTHORTWO}
\fancyfoot[CO]{\thepage}
\renewcommand\headrulewidth{0.5pt}
\renewcommand\footrulewidth{0.5pt}
%\headsep = -2pt
%\footskip = 0pt

% define color
\definecolor{sectionColor}{HTML}{000000}
\definecolor{subSectionColor}{HTML}{000000}
\definecolor{subSubSectionColor}{HTML}{000000}
\definecolor{codeBackground}{RGB}{245,245,245}
\definecolor{gray}{rgb}{0.5,0.5,0.5}
\definecolor{darkGreen}{RGB}{0,150,0}
\definecolor{OSTPink}{RGB}{140,25,95}
\definecolor{DarkPurple}{rgb}{0.4, 0.1, 0.4}

% define section format
%define section color and size
\addtokomafont{section}{\color{sectionColor}}
\addtokomafont{subsection}{\color{subSectionColor}}
\addtokomafont{subsubsection}{ \color{subSubSectionColor}}
\addtokomafont{paragraph}{\small \color{subSubSectionColor}}
\addtokomafont{subparagraph}{\small \bfseries \itshape \color{subSubSectionColor}}

\makeatletter
\renewcommand\paragraph{\@startsection{paragraph}{4}{\z@}%
    {-3.25ex\@plus -1ex \@minus -.2ex}%
    {1.5ex \@plus .2ex}%
    {\normalfont\normalsize\itshape\bfseries}}

\renewcommand\subparagraph{\@startsection{subparagraph}{4}{\z@}%
    {-3.25ex\@plus -1ex \@minus -.2ex}%
    {1.5ex \@plus .2ex}%
    {\normalfont\normalsize\bfseries\itshape\color{black}}}
\makeatother

% new section -> new page
\let\stdchapter\chapter
\renewcommand\chapter{\clearpage\stdchapter}

% import code listings
\usepackage{listings}
\usepackage{beramono}
%! Licence = CC BY-NC-SA 4.0

%! Author = mariuszindel
%! Date = 22. Feb 2021
%! Project = latex-documentation-template


\lstdefinestyle{Java}{
    language=java,
    backgroundcolor = \color{codeBackground},       %color for the background
    basicstyle=\ttfamily\scriptsize,                % font size/family/etc. for source
    keywordstyle=\color{RoyalBlue}\ttfamily,        % style of keywords in source language
    stringstyle=\color{darkGreen}\ttfamily,         % style of strings in source language
    commentstyle=\color{DarkPurple!60}\ttfamily,    % style of comments in source language
    escapeinside={£}{£},                            % specify characters to escape from source code to LATEX
    showspaces=false,                               % emphasize spaces in code (true/false)
    showstringspaces=false,
    showtabs=false,                                 % emphasize tabulators in code (true/false)
    numbers=left,                                   % position of line numbers (left/right/none)
    numberstyle=\tiny\color{darkgray}\ttfamily,     % style used for line-numbers
    stepnumber=1,                                   % distance of line-numbers from the code
    tabsize=1,                                      % default tabsize
    breaklines=true,                                % automatic line-breaking
    breakatwhitespace=true,                         % sets if automatic breaks should only happen at whitespaces
    frame=single,                                   % showing frame outside code (none/leftline/topline/bottomline/lines/single/shadowbox)
    xleftmargin=15pt,
    xrightmargin=15pt,
    frameround=tttt,                            % enable round corners
    rulecolor = \color{lightgray},              % Specify the colour of the frame-box
    aboveskip = 6pt,
    belowskip = 15pt,
    captionpos = b                              % position of caption (t/b)
}

\lstdefinestyle{CSharp}{
    language=[Sharp]C,
    backgroundcolor = \color{codeBackground},       %color for the background
    basicstyle=\ttfamily\scriptsize,                % font size/family/etc. for source
    keywordstyle=\color{RoyalBlue}\ttfamily,        % style of keywords in source language
    stringstyle=\color{darkGreen}\ttfamily,         % style of strings in source language
    commentstyle=\color{DarkPurple!60}\ttfamily,    % style of comments in source language
    escapeinside={£}{£},                            % specify characters to escape from source code to LATEX
    showspaces=false,                               % emphasize spaces in code (true/false)
    showstringspaces=false,
    showtabs=false,                                 % emphasize tabulators in code (true/false)
    numbers=left,                                   % position of line numbers (left/right/none)
    numberstyle=\tiny\color{darkgray}\ttfamily,     % style used for line-numbers
    stepnumber=1,                                   % distance of line-numbers from the code
    tabsize=1,                                      % default tabsize
    breaklines=true,                                % automatic line-breaking
    breakatwhitespace=true,                         % sets if automatic breaks should only happen at whitespaces
    frame=single,                                   % showing frame outside code (none/leftline/topline/bottomline/lines/single/shadowbox)
    xleftmargin=15pt,
    xrightmargin=15pt,
    frameround=tttt,                            % enable round corners
    rulecolor = \color{lightgray},              % Specify the colour of the frame-box
    aboveskip = 6pt,
    belowskip = 15pt,
    captionpos = b                              % position of caption (t/b)
}

\lstdefinestyle{JavaScript}{
    keywords={typeof, new, true, false, catch, function, return, null, catch, switch, var, if, in, while, do, else, case, break, const},
    ndkeywords={class, export, boolean, throw, implements, import, this},
    comment=[l]{//},
    backgroundcolor = \color{codeBackground},       %color for the background
    basicstyle=\ttfamily\scriptsize,                % font size/family/etc. for source
    keywordstyle=\color{RoyalBlue}\ttfamily,        % style of keywords in source language
    stringstyle=\color{darkGreen}\ttfamily,         % style of strings in source language
    commentstyle=\color{DarkPurple!60}\ttfamily,    % style of comments in source language
    escapeinside={£}{£},                            % specify characters to escape from source code to LATEX
    showspaces=false,                               % emphasize spaces in code (true/false)
    showstringspaces=false,
    showtabs=false,                                 % emphasize tabulators in code (true/false)
    numbers=left,                                   % position of line numbers (left/right/none)
    numberstyle=\tiny\color{darkgray}\ttfamily,     % style used for line-numbers
    stepnumber=1,                                   % distance of line-numbers from the code
    tabsize=1,                                      % default tabsize
    breaklines=true,                                % automatic line-breaking
    breakatwhitespace=true,                         % sets if automatic breaks should only happen at whitespaces
    frame=single,                                   % showing frame outside code (none/leftline/topline/bottomline/lines/single/shadowbox)
    xleftmargin=15pt,
    xrightmargin=15pt,
    frameround=tttt,                            % enable round corners
    rulecolor = \color{lightgray},              % Specify the colour of the frame-box
    aboveskip = 6pt,
    belowskip = 15pt,
    captionpos = b                              % position of caption (t/b)
}

\lstdefinestyle{GithubActionsYAML}{
    keywords={true,false,null,y,n},
    ndkeywords={on, jobs, env, name, runs-on},
    comment=[l]{//},
    backgroundcolor = \color{codeBackground},       %color for the background
    basicstyle=\ttfamily\scriptsize,                % font size/family/etc. for source
    keywordstyle=\color{RoyalBlue}\ttfamily,        % style of keywords in source language
    stringstyle=\color{darkGreen}\ttfamily,         % style of strings in source language
    commentstyle=\color{DarkPurple!60}\ttfamily,    % style of comments in source language
    escapeinside={£}{£},                            % specify characters to escape from source code to LATEX
    showspaces=false,                               % emphasize spaces in code (true/false)
    showstringspaces=false,
    showtabs=false,                                 % emphasize tabulators in code (true/false)
    numbers=left,                                   % position of line numbers (left/right/none)
    numberstyle=\tiny\color{darkgray}\ttfamily,     % style used for line-numbers
    stepnumber=1,                                   % distance of line-numbers from the code
    tabsize=1,                                      % default tabsize
    breaklines=true,                                % automatic line-breaking
    breakatwhitespace=true,                         % sets if automatic breaks should only happen at whitespaces
    frame=single,                                   % showing frame outside code (none/leftline/topline/bottomline/lines/single/shadowbox)
    xleftmargin=15pt,
    xrightmargin=15pt,
    frameround=tttt,                            % enable round corners
    rulecolor = \color{lightgray},              % Specify the colour of the frame-box
    aboveskip = 6pt,
    belowskip = 15pt,
    captionpos = b                              % position of caption (t/b)
}

\lstdefinestyle{HTML}{
    language=HTML,
    backgroundcolor = \color{codeBackground},       %color for the background
    basicstyle=\ttfamily\scriptsize,                % font size/family/etc. for source
    keywordstyle=\color{RoyalBlue}\ttfamily,        % style of keywords in source language
    stringstyle=\color{darkGreen}\ttfamily,         % style of strings in source language
    commentstyle=\color{DarkPurple!60}\ttfamily,    % style of comments in source language
    escapeinside={£}{£},                            % specify characters to escape from source code to LATEX
    showspaces=false,                               % emphasize spaces in code (true/false)
    showstringspaces=false,
    showtabs=false,                                 % emphasize tabulators in code (true/false)
    numbers=left,                                   % position of line numbers (left/right/none)
    numberstyle=\tiny\color{darkgray}\ttfamily,     % style used for line-numbers
    stepnumber=1,                                   % distance of line-numbers from the code
    tabsize=1,                                      % default tabsize
    breaklines=true,                                % automatic line-breaking
    breakatwhitespace=true,                         % sets if automatic breaks should only happen at whitespaces
    frame=single,                                   % showing frame outside code (none/leftline/topline/bottomline/lines/single/shadowbox)
    xleftmargin=15pt,
    xrightmargin=15pt,
    frameround=tttt,                            % enable round corners
    rulecolor = \color{lightgray},              % Specify the colour of the frame-box
    aboveskip = 6pt,
    belowskip = 15pt,
    captionpos = b                              % position of caption (t/b)
}

\lstdefinestyle{Python}{
    language=Python,
    backgroundcolor = \color{codeBackground},       %color for the background
    basicstyle=\ttfamily\scriptsize,                % font size/family/etc. for source
    keywordstyle=\color{RoyalBlue}\ttfamily,        % style of keywords in source language
    stringstyle=\color{darkGreen}\ttfamily,         % style of strings in source language
    commentstyle=\color{DarkPurple!60}\ttfamily,    % style of comments in source language
    escapeinside={£}{£},                            % specify characters to escape from source code to LATEX
    showspaces=false,                               % emphasize spaces in code (true/false)
    showstringspaces=false,
    showtabs=false,                                 % emphasize tabulators in code (true/false)
    numbers=left,                                   % position of line numbers (left/right/none)
    numberstyle=\tiny\color{darkgray}\ttfamily,     % style used for line-numbers
    stepnumber=1,                                   % distance of line-numbers from the code
    tabsize=1,                                      % default tabsize
    breaklines=true,                                % automatic line-breaking
    breakatwhitespace=true,                         % sets if automatic breaks should only happen at whitespaces
    frame=single,                                   % showing frame outside code (none/leftline/topline/bottomline/lines/single/shadowbox)
    xleftmargin=15pt,
    xrightmargin=15pt,
    frameround=tttt,                            % enable round corners
    rulecolor = \color{lightgray},              % Specify the colour of the frame-box
    aboveskip = 6pt,
    belowskip = 15pt,
    captionpos = b                              % position of caption (t/b)
}

\lstdefinestyle{bash}{
    language=bash,
    backgroundcolor = \color{codeBackground},       %color for the background
    basicstyle=\ttfamily\scriptsize,                % font size/family/etc. for source
    keywordstyle=\color{RoyalBlue}\ttfamily,        % style of keywords in source language
    stringstyle=\color{darkGreen}\ttfamily,         % style of strings in source language
    commentstyle=\color{DarkPurple!60}\ttfamily,    % style of comments in source language
    escapeinside={£}{£},                            % specify characters to escape from source code to LATEX
    showspaces=false,                               % emphasize spaces in code (true/false)
    showstringspaces=false,
    showtabs=false,                                 % emphasize tabulators in code (true/false)
    numbers=left,                                   % position of line numbers (left/right/none)
    numberstyle=\tiny\color{darkgray}\ttfamily,     % style used for line-numbers
    stepnumber=1,                                   % distance of line-numbers from the code
    tabsize=1,                                      % default tabsize
    breaklines=true,                                % automatic line-breaking
    breakatwhitespace=true,                         % sets if automatic breaks should only happen at whitespaces
    frame=single,                                   % showing frame outside code (none/leftline/topline/bottomline/lines/single/shadowbox)
    xleftmargin=15pt,
    xrightmargin=15pt,
    frameround=tttt,                            % enable round corners
    rulecolor = \color{lightgray},              % Specify the colour of the frame-box
    aboveskip = 6pt,
    belowskip = 15pt,
    captionpos = b                              % position of caption (t/b)
}

\lstdefinestyle{LaTeX}{
    language=TeX,
    backgroundcolor = \color{codeBackground},       %color for the background
    basicstyle=\ttfamily\scriptsize,                % font size/family/etc. for source
    keywordstyle=\color{RoyalBlue}\ttfamily,        % style of keywords in source language
    stringstyle=\color{darkGreen}\ttfamily,         % style of strings in source language
    commentstyle=\color{DarkPurple!60}\ttfamily,    % style of comments in source language
    escapeinside={£}{£},                            % specify characters to escape from source code to LATEX
    showspaces=false,                               % emphasize spaces in code (true/false)
    showstringspaces=false,
    showtabs=false,                                 % emphasize tabulators in code (true/false)
    numbers=left,                                   % position of line numbers (left/right/none)
    numberstyle=\tiny\color{darkgray}\ttfamily,     % style used for line-numbers
    stepnumber=1,                                   % distance of line-numbers from the code
    tabsize=1,                                      % default tabsize
    breaklines=true,                                % automatic line-breaking
    breakatwhitespace=true,                         % sets if automatic breaks should only happen at whitespaces
    frame=single,                                   % showing frame outside code (none/leftline/topline/bottomline/lines/single/shadowbox)
    xleftmargin=15pt,
    xrightmargin=15pt,
    frameround=tttt,                            % enable round corners
    rulecolor = \color{lightgray},              % Specify the colour of the frame-box
    aboveskip = 6pt,
    belowskip = 15pt,
    captionpos = b                              % position of caption (t/b)
}
\lstset{style=bash}
\lstset{
literate=  % Allow for German characters in lstlistings.
    {Ö}{{\"O}}1
{Ä}{{\"A}}1
{Ü}{{\"U}}1
{ü}{{\"u}}1
{ä}{{\"a}}1
{ö}{{\"o}}1
}

% dotted rule
\usepackage{dashrule}
\usepackage{tikz}
\usetikzlibrary{decorations.markings}
\newcommand{\drule}[3][0]{
    \tikz[baseline]{\path[decoration={markings,
                    mark=between positions 0 and 1 step 2*#3
                    with {\node[fill, circle, minimum width=#3, inner sep=0pt, anchor=south west] {};}},postaction={decorate}]  (0,#1) -- ++(#2,0);}}

% no indentation
\setlength{\parindent}{0cm}

% include lorem ipsum
\usepackage{lipsum}

% degree symbol
\usepackage{textcomp}

% insert PDF's
\usepackage{pdfpages}

% change layout to landscape for one page
\usepackage{pdflscape}

% globaly small URL
\renewcommand{\UrlFont}{\ttfamily\small}

% Center p element in table
\newcolumntype{P}[1]{>{\centering\arraybackslash}p{#1}}

\usepackage[backend=biber,style=alphabetic,sorting=ynt]{biblatex}
\addbibresource{../bibliography.bib}

\usepackage[acronyms, toc]{glossaries}
\makeglossaries

% in this file
% \newglossaryentry{latex}
% {
%     name=latex,
%     description={Is a arkup language specially suited 
%     for scientific documents}
% }


% In text
% \Gls{latex}

\newglossaryentry{identity-access-management}
{
    name=Identity and Access Management,
    description={Identitäts- und Zugriffsmanagement (IAM) können Administratoren autorisieren, wer auf bestimmte Ressourcen zugreifen darf.
            So ist es möglich die Kontrolle und Transparenz zentral zu verwalten.
            Für Unternehmen mit komplexen Organisationsstrukturen, Hunderten von Teams und vielen Projekten bietet IAM eine einheitliche Sicht auf die Sicherheitsrichtlinien in Ihrem gesamten Unternehmen mit integrierter Prüfung zur Vereinfachung der Compliance-Prozesse}
}

\newglossaryentry{azure-active-directory-domain-services}
{
    name=Azure Active Directory Domain Services,
    description={Azure Active Directory (Azure AD) ist ein Cloud-basierter Identitäts- und Zugangsverwaltungsdienst.
            Dieser Dienst hilft beim Zugriff auf externe Ressourcen, wie Microsoft 365, das Azure-Portal und Tausende anderer SaaS-Anwendungen}
}

%CA, NTP, DNS, DHCP
\newglossaryentry{network-time-protocol}
{
    name=Network Time Protocol,
    description={Das Network Time Protocol (NTP) wird häufig zur Synchronisierung von Computeruhren im Netzwerk verwendet}
}

\newglossaryentry{certification-authority}
{
    name=Certification Authority,
    description={Die Certification Authoritity (CA) ist eine Stelle, welche digitale Zertifikate ausstellt. Somit ist es möglich bei der Kommunikation zweier Parteien die Integrität durch zuverlässige dritte Partei zu haben}
}

\newglossaryentry{dynamic-host-configuration-protocol}
{
    name=Dynamic Host Configuration Protocol,
    description={Das Dynamic Host Configuration Protocol (DHCP) ist ein Protokoll im Netzwerk. Es ermöglicht die Zuweisung von Netzwerkkonfigurationen, wie IP-Adressen \& Gateway, an Clients durch einen Server}
}

\newglossaryentry{domain-name-system}
{
    name=Domain Name System,
    description={Das Domain Name System (DNS) ist das Telefonbuch eines IP-basierten Netzwerkes. Seine Hauptaufgabe ist die Beantwortung von Anfragen zur Namensauflösung}
}

\newglossaryentry{national-institute-of-standards-and-technology}
{
    name=National Institute of Standards and Technology,
    description={Azure Active Directory (Azure AD) ist ein Cloud-basierter Identitäts- und Zugangsverwaltungsdienst.
            Dieser Dienst hilft beim Zugriff auf externe Ressourcen, wie Microsoft 365, das Azure-Portal und Tausende anderer SaaS-Anwendungen}
}

\newglossaryentry{trusted-platform-module}
{
    name=Trusted Platform Module,
    description={
            Das Trusted Platform Module ist ein optionaler Hardware-Chip auf der Hauptplatine in einem Computer. IM TPM Chip werden Kryptografische Schlüssel hinterlegt, welche dann von Software verwendet werden können.
        }
}
%! Licence = CC BY-NC-SA 4.0

%! Author = severingrimm
%! Date = 04.10.2021
%! Project = SA HS 2021 - NUTS

% in text
% \acrlong{ }
% Displays the phrase which the acronyms stands for. Put the label of the acronym inside the braces. In the example, \acrlong{gcd} prints Greatest Common Divisor.
% \acrshort{ }
% Prints the acronym whose label is passed as parameter. For instance, \acrshort{gcd} renders as GCD.
% \acrfull{ }
% Prints both, the acronym and its definition. In the example the output of \acrfull{lcm} is Least Common Multiple (LCM).

% use in abbreviations.tex
%\newacronym{cd}{CD}{Continuous Deployment}


\newacronym{iam}{IAM}{\Gls{identity-access-management}}
\newacronym{adds}{ADDS}{\Gls{azure-active-directory-domain-services}}
\newacronym{ntp}{NTP}{\Gls{network-time-protocol}}
\newacronym{ca}{CA}{\Gls{certification-authority}}
\newacronym{dns}{DNS}{\Gls{certification-authority}}
\newacronym{dhcp}{DHCP}{\Gls{dynamic-host-configuration-protocol}}
\newacronym{nist}{NIST}{\Gls{national-institute-of-standards-and-technology}}
\newacronym{sans}{SANS}{\Gls{sans-insitute}}
\newacronym{tpm}{TPM}{\Gls{trusted-platform-module}}
\newacronym{wsus}{WSUS}{\Gls{windows-server-update-services}}
\newacronym{cve}{CVE}{\Gls{common-vulnerabilities-and-exposures}}
\newacronym{av}{AV}{\Gls{antivirus}}
\newacronym{rto}{RTO}{\Gls{recovery-time-objective}}
\newacronym{ou}{OU}{\Gls{organizational-unit}}
\newacronym{gpo}{GPO}{\Gls{group-policy-object}}
\newacronym{laps}{LAPS}{\Gls{local-administrator-password-solution}}
\newacronym{os}{OS}{\Gls{operating-system}}
\newacronym{siem}{SIEM}{\Gls{security-information-and-event-management}}
\newacronym{elk}{ELK}{\Gls{elasticsearch-logstash-kibana}}
\newacronym{gui}{GUI}{\Gls{graphical-user-interface}}
\newacronym{kql}{KQL}{\Gls{kibana-query-language}}
