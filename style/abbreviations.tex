%! Licence = CC BY-NC-SA 4.0

%! Author = severingrimm
%! Date = 04.10.2021
%! Project = SA HS 2021 - NUTS

% in text
% \acrlong{ }
% Displays the phrase which the acronyms stands for. Put the label of the acronym inside the braces. In the example, \acrlong{gcd} prints Greatest Common Divisor.
% \acrshort{ }
% Prints the acronym whose label is passed as parameter. For instance, \acrshort{gcd} renders as GCD.
% \acrfull{ }
% Prints both, the acronym and its definition. In the example the output of \acrfull{lcm} is Least Common Multiple (LCM).

% use in abbreviations.tex
%\newacronym{cd}{CD}{Continuous Deployment}


\newacronym{iam}{IAM}{\Gls{identity-access-management}}
\newacronym{adds}{ADDS}{\Gls{azure-active-directory-domain-services}}
\newacronym{ntp}{NTP}{\Gls{network-time-protocol}}
\newacronym{ca}{CA}{\Gls{certification-authority}}
\newacronym{dns}{DNS}{\Gls{certification-authority}}
\newacronym{dhcp}{DHCP}{\Gls{dynamic-host-configuration-protocol}}
\newacronym{nist}{NIST}{\Gls{national-institute-of-standards-and-technology}}
\newacronym{tpm}{TPM}{\Gls{trusted-platform-module}}