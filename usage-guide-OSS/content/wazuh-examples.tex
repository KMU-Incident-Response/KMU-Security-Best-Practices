\chapter{Incident Erkennung}
\section{Einleitung}
%TODO: Hier gibt es Beispiele, wichtig für erkennung. Log dateien anschauen
%Wichtig meistens Level 12 Alerts. 

\section{Queries}
%IMAGE FILTER

\section{Attacken Beispiele}
\subsection{Mimikatz}
%Mimikatz ist Tool für auslesen von passwort hash gespeichert auf PC. Mit diesen Hash kann pass the hash attacken gemacht werden. Dabei wird PW nicht gebraucht. 

%Spezifische Reaktion: Betroffene Domain Admin Passwörter wechseln.  

%Mitigation: LAPS, möglichst wenige accounts mit Admin Rechten auf  

\subsection{Brute Force}
%Bruteforce attacken

%Spezifische Reaktion: Welcher Account, welcher PC. Wo ist PC. Brute force über RDP, Physisch am Geräte, SSH. Sicherstellen, dass ein gutes PW gesetzt ist.

%Mitigation: Gute Passwortrichtlinien, starche PWs


\section{False Positives}
\subsection{Check Hash}
%Attacker benutzten gerne bekannte Namen, Hash ist aber immer eindeutig. Wenn nicht gleiche Datei, anderer Hash
%IMAGE CHECK-HASH-1
%Check parentimage that ran svchost. check hash of svchost of file with SAME Version (Windows)
%Optional, check Parent Image Hash
%certutil -hashfile "C:\Windows\System32\svchost.exe" SHA256
%Hash: f3feb95e7bcfb0766a694d93fca29eda7e2ca977c2395b4be75242814eb6d881

%VirusTotal verlinken

\section{IT-Dienstleister}