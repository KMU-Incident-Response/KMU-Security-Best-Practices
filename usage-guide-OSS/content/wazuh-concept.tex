\chapter{Wazuh Übersicht}
\section{Entstehung}
Wazuh ist aus dem Open-Source Projekt \href{https://github.com/ossec}{ossec}\footnote{Link: https://github.com/ossec} entstanden.
Wazuh ist wie ossec komplett Open-Source und kostenlos verfügbar.

\section{Aufbau}
Wazuh basiert auf dem \acrfull{elk} Stack. 
Es ist ein Plugin, welches in den \acrshort{elk} Stack integriert werden kann und mithilfe von Agents auf den Computern Logdateien sammelt.

\begin{figure}[H]
    \centering
    \includegraphics[width=0.5\linewidth]{../img/aufbau-wazuh.png}
    \caption[Übersicht Wazuh]{Übersicht Wazuh\footnotemark}
\end{figure}
\footnotetext{Zugriff: 24.04.2022 \cite{wazuh-documentation}}


%TODO: maybe say deepdive installer handels most for starters

\subsection{Wazuh Manager}
Der Wazuh Manager wird auf einem Linux Server installiert.
Der Wazuh Manager beinhaltet den kompletten \acrshort{elk} Stack, mit Wazuh Plugin. 

\subsubsection{ossec.conf}
In der ossec.conf Datei sind alle Konfigurationen vom Wazuh-Manager abgespeichert.

%TODO: path maybe?


\subsection{Wazuh Agent}
Der Wazuh Agent wird auf dem Client installiert.
Es werden die meisten Betriebssysteme unterstützt. Darunter viele Linux Systeme, Windows und MacOS.\\

Der Wazuh Agent leitet alle Logdateien, die in der agent.conf definiert sind, weiter an den Wazuh Manager.
Zusätzlich überwacht der Wazuh Agent auch alle Systemdatei- und Registryänderungen und leitet diese an den Wazuh Manager weiter. 

\subsection{Ablauf}
Die Agents senden die Logs an den Wazuh Manager oder der Wazuh Manager holt die Logs mit rsyslog von den Systemen.
%TODO: pretty sure this isn't how rsyslog work 
%TODO: Logeintrag passender?
Danach werden die Logs mit einem Decoder strukturiert und die Regeln werden auf die Logs angewendet.
Wenn eine Regel zutrifft, wird ein Alert generiert und angezeigt.
Wenn keine Regel zutrifft, wird das Log verworfen.
%TODO: Logeintrag passender?
In der ossec.conf kann eingestellt werden, dass auch die Logs abgespeichert werden, die auf keine Regel zutreffen.
%TODO: Logeintrag passender?
Dabei sammeln sich grosse Datenmengen an und dies ist hauptsächlich für Debugging empfohlen.

\begin{figure}[H]
    \centering
    \includegraphics[width=\linewidth]{../img/wazuh-ablauf.png}
    \caption[Wazuh Ablauf]{Wazuh Ablauf\footnotemark}
\end{figure}
\footnotetext{Zugriff: 24.04.2022 \cite{wazuh-documentation}}

\section{Rules}
\subsection{Alert Level}
Es gibt Alert Levels von 0 bis 16. 
Diese bedeuten nicht wie schwerwiegend ein Ereignis ist, sondern jedes Level hat eine spezielle Bedeutung.
Die wichtigsten Level sind folgende:
\begin{itemize}
    \item \textbf{Level 3} sind Alerts, welche authorisiert sind und tendenziell nicht gefährlich.
    \item \textbf{Level 12} sind Alerts, welche eine Anomalie darstellen und potenziell gefährlich sind.
\end{itemize}

Alle Alert Level findet man in der \href{https://documentation.wazuh.com/current/user-manual/ruleset/rules-classification.html}{Wazuh Dokumentation}\footnote{Link: https://documentation.wazuh.com/current/user-manual/ruleset/rules-classification.html}.

\section{Decoders}
Da Logs in allen möglichen Formen daherkommen, je nach Herkunft braucht es verschiedene Decoder.
Die Decoder bringen die eingehenden Logs in eine einheitliche Struktur, um die Regeln darauf anwenden zu können.
Es werden Decoder für einige bekannten Logformate angeboten, wie zum Beispiel für den Windows Event Manager oder Cisco IOS Logs.


\section{Groups}
Die Gruppen werden verwendet, um ähnliche Geräte zu gruppieren.
Für jede Gruppe kann man eine agent.conf einrichten, in welcher eingetragen wird welche Logdateien von diesen Systemen verarbeitet werden sollen. 

\subsubsection{agent.conf}
In der agent.conf kann definiert werden, welche Logdateien an den Wazuh Manager weitergeleitet werden. 
Eine Lokation wird mit <localfile> angegeben:
\begin{lstlisting}
<agent_config os="Windows">
    <localfile>
        <location>Microsoft-Windows-Sysmon/Operational</location>
        <log_format>eventchannel</log_format>
    </localfile>
</agent_config>
\end{lstlisting}
Weitere Informationen wie neue Orte mit Logdateien eingebunden werden können, findet man in der \href{https://documentation.wazuh.com/current/user-manual/reference/centralized-configuration.html?highlight=agent%20conf}{Wazuh Dokumentation}\footnote{Link: https://documentation.wazuh.com/current/user-manual/reference/centralized-configuration.html?highlight=agent\%20conf}