\section{Incident Response Prozess}
Der Incident Response Prozess des \acrfull{sans} definiert die Vorbereitung und den Ablauf eines Incidents, sowie die Wiederherstellung nach einem Incident.
Dies wird in sechs Schritte aufgeteilt. 

\subsection{Prepare}


In dieser GitHub Organisation wird der ``Prepare'' Teil durch das Incident Response Plan Template, die Installation von Wazuh und den Security Best Practices abgedeckt.

\subsection{Identify}

In dieser GitHub Organisation wird der ``Identify'' Teil durch die Erklärungen und Beispiele von Wazuh abgedeckt.

\subsection{Contain}



\subsection{Eradicate}


\subsection{Recover}


\subsection{Lessons learned}



%Incident Response prozess. (NIST)
%Containment
%Incident Response Plan Prozess starten.
%IT Dienstleister informieren wenn sicher