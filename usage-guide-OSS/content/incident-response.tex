\section{Incident Response Prozess}
Der Incident Response Prozess des \acrfull{sans} definiert die Vorbereitung und den Ablauf eines Incidents, sowie die Wiederherstellung nach einem Incident.
Dieser wird in sechs Schritte aufgeteilt. 

\subsection{Prepare}
In der Phase ``Prepare'' wird ein Unternehmen auf einen Incident vorbereitet. 
Dazu gehört die technische, wie auch die organisatorische Vorbereitung.\\

In der technischen Vorbereitung werden Unternehmen mit Sicherheitsmassnahmen, Software und Hardware geschützt.
Ausserdem wird sichergestellt, dass das Incident Response Team die nötigen Berechtigungen hat, um einen Incident behandeln zu können.\\

In der organisatiorischen Vorbereitung werden Richtlinien definiert, was in einem Unternehmen erlaubt ist im Umgang mit der IT Infrastruktur und wo Einschränkungen getroffen werden.
Es wird ein Incident Response Plan erstellt, welcher den Ablauf im Falle eines Incidents regelt.
Dabei wird auch das Vorgehen in der Kommunikation definiert und und Notfallkontakte definiert.\\

In dieser GitHub Organisation\footnote{Link: \href{https://github.com/KMU-Incident-Response}{https://github.com/KMU-Incident-Response}} wird der ``Prepare'' Teil durch das Incident Response Plan Template, die Installation von Wazuh und den Security Best Practices abgedeckt.

\subsection{Identify}
In der Phase ``Identify'' wird geregelt, wie Incidents erkannt werden. Nachfolgend kann der vorhergehend definierte Plan gestartet werden.
Dies kann über Mitarbeitende passieren, die einen Incident dem IT Support melden, über ein automatisches Intrusion Detection System oder über Alerts von einem \acrshort{siem}.
Falls ein Incident festgestellt wird, sollte der externe IT Dienstleister umgehend informiert werden.\\

In dieser GitHub Organisation\footnote{Link: \href{https://github.com/KMU-Incident-Response}{https://github.com/KMU-Incident-Response}} wird der ``Identify'' Teil durch die Erklärungen und Beispiele von Wazuh abgedeckt.

\subsection{Contain}
Die ``Containment'' Phase beinhaltet die Eindämmung eines Incidents, um weitere Folgeschäden zu vermeiden.
Diese Phase ist in drei Schritte aufgeteilt. 
Das ``short-term Containment'', das ``System Backup'' und das ``long-term Containment''.\\

Im ``short-term Containment'' ist der Fokus, den Schaden möglichst schnell einzudämmen.
Dies kann unter anderem durch trennen der betroffenen System vom Netzwerk sein, betroffene Systeme herunterfahren und/oder Zugriffsrechte von Benutzeraccounts einzuschränken.
Dieser Teil kann oftmals direkt vom IT Support ausgeführt werden, bevor ein externer IT Dienstleister bereit ist einzugreifen.\\

Im ``System Backup'' Teil werden forensische Abbildungen der betroffenen Systeme erstellt.
Mit diesen kann in der ``Lessons learned'' Phase nachvollzogen werden, wie Angreifer in die Systeme eingedrungen sind und wo allfällige Schwachstellen sind. 
Für forensische Abbildungen werden spezifische Tools benötigt, welche ein KMU nicht besitzt. 
Daher sollte dieser Schritt vom externen IT Dienstleister durchgeführt werden.

Im ``long-term Containment'' werden die betroffenen Systeme soweit gesichert, dass diese wieder vorübergehend Produktuionsfähig sind, bevor in der nächsten Phase ``Eradicate'' saubere Systeme aufgesetzt werden.

\subsection{Eradicate}
Die ``Eradicate'' Phase behandelt die eigentliche Entfernung aller kompromittierten Systeme und Malware im Netzwerk.
Bevor die Systeme wiederhergestellt werden, muss sichergestellt werden, dass alle Eintrittspunkte der Angreifer blockiert wurden und keine Malware sich im Netzwerk mehr befindet.
Sonst werden neue Systeme direkt wieder kompromittiert.

\subsection{Recover}
In der ``Recover'' Phase werden die betroffenen Systeme wiederhergestellt.
Es ist wichtig, die neuen Systeme genau zu überwachen um sicherzugehen, dass diese nicht direkt wieder kompromittiert werden. 


\subsection{Lessons learned}
In der ``Lessons learned'' Phase werden alle Dokumentationen fertiggestellt, alle Erkentnisse aus dem Incident zusammengefasst und ein Report erstellt.
Dies soll für weitere Incidents helfen, schneller reagieren zu können oder als Fallbeispiel für neue Mitarbeitende zu Verfügung stehen. 