\chapter{Sysmon}
\section{Übersicht}
System Monitor (Sysmon) ist ein Windows-Systemdienst und -Gerätetreiber, der, sobald er auf einem System installiert ist, bei jedem Neustart des Systems aktiv bleibt, um die Systemaktivitäten zu überwachen und im Windows-Ereignisprotokoll zu protokollieren. Er liefert detaillierte Informationen über die Erstellung von Prozessen, Netzwerkverbindungen und Änderungen der Dateierstellungszeit. Durch das Sammeln der von ihm erzeugten Ereignisse mithilfe der Windows-Ereignissammlung oder SIEM-Agenten und die anschließende Analyse dieser Ereignisse können Sie bösartige oder anomale Aktivitäten identifizieren und verstehen, wie Eindringlinge und Malware in Ihrem Netzwerk operieren.\\

Beachten Sie, dass Sysmon keine Analyse der von ihm erzeugten Ereignisse bereitstellt und auch nicht versucht, sich vor Angreifern zu schützen oder zu verstecken.\footnote{Übersetzt mit www.DeepL.com/Translator: \href{https://docs.microsoft.com/en-us/sysinternals/downloads/sysmon}{https://docs.microsoft.com/en-us/sysinternals/downloads/sysmon}}

\section{Konfiguration}
Sysmon alleine macht nichts. 
Erst mit einer Konfiguration, in welcher definiert wird was alles in den Eventlog geschrieben wird, kann Sysmon verwendet werden.
Diese Konfiguration ist im XML Format. \\

Wichtig ist zu wissen, dass Sysmon eventuell auch businesskritische oder vertrauliche Daten von Prozessen loggt und diese dann über den Wazuh Agent an den Manager gesendet werden.
Dies kann datenschutztechnische Probleme geben oder vertrauliche Daten können von unbefugten Mitarbeitenden im Wazuh GUI gesehen werden.
Daher sollte die Konfiguration immer zuerst den Anforderungen entsprechend angepasst und überprüft werden.
Proprietäre Software kann auch vom Sysmon Logging ausgeschlossen werden.
