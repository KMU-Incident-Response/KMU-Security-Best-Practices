\chapter{Wazuh Agent Installation}

\section{Installation via GPO}

%1. Download Wazuh Agent
%https://documentation.wazuh.com/current/installation-guide/packages-list.html#windows
%64bit if only 64bit system else 32bit

%2.5 Download Sysmon
%https://docs.microsoft.com/en-us/sysinternals/downloads/sysmon
%
%2. Download Windows SDK
%https://documentation.wazuh.com/current/installation-guide/packages-list.html#windows
%
%3. Install only MSI Tools [Install-SDK]
%
%4. Go to C:\Program Files (x86)\Windows Kits\10\bin\<Version>\x86 and run Orca
%
%5. Right Klick Wazuh MSI and select Edit with Orca
%

%6. Select Transform > New Transform

%7. Go to Property and add key value pairs: [Orca-Edit]

%WAZUH_MANAGER: <Hostname>
%WAZUH_REGISTRATION_SERVER: <Hostname>
%WAZUH_AGENT_GROUP: Windows
%WAZUH_REGISTRATION_PASSWORD: <Set Password>

%8. Select Transform > Generate Transform
%Save as custom-wazuh-settings.mst

%9. Copy to Server to path every client has access to. E.g.: C:\Windows\SYSVOL\sysvol\ba.lab\Applications

%10. Create new Group Policy on OU of Windows Clients/Servers. [edit-new-group-policy]

%11. Go to Computer Configuration > Policies Software Settings > Software Installation > New > Package [new-software-install.png]

%12. Window Opens. Select .msi file. New windows opens Select Advanced
%Under modifications select add and select .mst file [new-software-install-advanced-mod.png]

%13 GPO is now ready. PC install it on restart


\section{Manuelle Installation}

Windows Agent kann mittels Powershell installiert werden.


1. Wazuh Webseite öffnen und unter Wazuh > Agent auf Deploy new Agent drücken. [Deploy-new-agent]

2. Daten auswählen [Deploy-new-agent-2] und PS Kopieren

3. Powershell auf Computer als Admin starten

4 Ausführen
\begin{lstlisting}
    Invoke-WebRequest -Uri https://packages.wazuh.com/4.x/windows/wazuh-agent-4.2.6-1.msi -OutFile wazuh-agent-4.2.6.msi; ./wazuh-agent-4.2.6.msi /q WAZUH_MANAGER='wazuh-server' WAZUH_REGISTRATION_SERVER='wazuh-server' WAZUH_REGISTRATION_PASSWORD=<Password> WAZUH_AGENT_GROUP='Windows' 
\end{lstlisting}

5. Done